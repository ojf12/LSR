\documentclass{article}

\usepackage[utf8]{inputenc}
\usepackage{mathtools}
\usepackage{amsmath}
\usepackage{amssymb}
\usepackage{amsthm}
\usepackage{booktabs}
\usepackage{multirow}
\usepackage{pgfgantt}
\usepackage{enumitem}
\usepackage{etoolbox}

\patchcmd{\thebibliography}{\section*{\refname}}{}{}{}


\title{Appendix to the HiPEDS Late Stage Review }
\author{Omar J. Faqir}
\date{September 2018}


\begin{document}

\maketitle

\section{Provisional Thesis Table of Contents}

\section{Planned Research to be Completed}

\begin{enumerate} [series=myouterlist]
    \item Extend data collection problem to self-triggered control subject to bounded disturbances \textit{(December 2018/January 2019)}. Include trajectory planning as triggering conditions \textit{(March 2019)}.
\end{enumerate}

\noindent We aim to apply principles of self-triggered control to the proposed data collection problem in order to determine node sleep/wake times. We start by considering bounded exogenous inputs (e.g. wind, incoming data) and compute triggering times to guarantee feasibility/performance metrics subject to the predetermined optimal control policy. This will be extended by incorporating triggering times as decision variables into the optimal control problem (OCP). The co-design of control inputs and self-triggering times has been known as ``minimum attention control'' \cite{}. 

\begin{enumerate}[resume=myouterlist]
    \item Perform feasibility and stability analysis of data collection problem, as posed in LSR poster \textit{(December 2018)}.
\end{enumerate}

\noindent We have already proposed an OCP formulation for determining energy-optimal data aggregation in a mobile network, accounting for mobility and transmission dynamics. Although closed-loop simulations have been performed, we have yet to determine the conditions under which the proposed formulation will be feasible and stable when implemented in a decreasing horizon approach.

\begin{enumerate}[resume=myouterlist]
    \item Integrate sequential quadratic programming solver (SQP) into ICLOCS2 \textit{(December 2018)}. Incorporate SQP solver into existing mesh refinement schemes  \textit{(April 2019)}. Investigate potential links between self/event-triggering conditions and mesh refinement scheme \textit{(October 2019)}.
\end{enumerate}

\noindent ICLOCS2 is a MATLAB based transcription and solution environment for formulating and solving continuous time OCPs. 

\begin{enumerate}[resume=myouterlist]
    \item Extend modelling of data collection problem, as posed in LSR poster \textit{(October 2019)}.
\end{enumerate}

\noindent We firstly would like to be able to model the effect of the aircraft banking angle on the communication channel. Secondly we would like to be able to incorporate a model for data compression, and an associated model for the computation energy required for this. Both of these models are present, to some degree, in existing literature. 

\begin{enumerate}[resume=myouterlist]
    \item Extension of self-triggering work.  
\end{enumerate}


\begin{ganttchart}[hgrid, vgrid,
newline shortcut=false,
bar label node/.append style=%
{align=right} ]{1}{21}

\gantttitle{2018}{3}
\gantttitle{2019}{12}
\gantttitle{2020}{6} \ganttnewline
\gantttitlelist{1,...,3}{1}
\gantttitlelist{1,...,12}{1} 
\gantttitlelist{1,...,6}{1} \ganttnewline
  
%Self-triggerin stuff
\ganttbar[name=t1]{1}{1}{4}  \ganttnewline

% Extension of self triggering work
\ganttbar[name=t1]{5}{4}{16} \ganttnewline
\ganttbar[name=t1]{5}{4}{16} \ganttnewline

% Feasibility and stability analysis of current problem
\ganttbar[name=t1]{2}{1}{3}  \ganttnewline

% SQP stuff
\ganttbar[name=t1]{3}{1}{8}  \ganttnewline

% Modeling stuff
\ganttbar[name=t1]{4}{8}{10} 
\end{ganttchart}



% \begin{ganttchart}[hgrid, vgrid,
% newline shortcut=false,
% bar label node/.append style=%
% {align=right} ]{1}{36}
%   % titles
%   \gantttitle{2017}{3} 
%   \gantttitle{2018}{12} 
%   \gantttitle{2019}{12}
%   \gantttitle{2020}{9}  \ganttnewline
%   \gantttitlelist{1,...,36}{1}  \ganttnewline
  
%   \ganttmilestone{CDC Conference}{3} 
%   \ganttmilestone{}{3} 
%   \ganttmilestone{}{15} 
%   \ganttmilestone{}{27} \ganttnewline
%   \ganttmilestone{Globecomm Conference}{14} 
%   \ganttmilestone{}{26} \ganttnewline
  
%   \ganttbar[name=t1]{Distributed OCP}{4}{32}  \ganttnewline
%   \ganttbar[name=t2]{System Modeling \& \\ Problem Formulation}{1}{16}  \ganttnewline
%   \ganttbar[name=t3]{Robust MPC}{8}{22}  \ganttnewline
%   \ganttbar[name=t4]{Hardware \& Co-design Problem}{22}{32}  \ganttnewline
%   \ganttbar[name=w1]{Writing and Review}{32}{36} \ganttnewline
  
  
%   % ICLOCS
%   \ganttgroup{\texttt{ICLOCS v2.1}}{1}{6}  \ganttnewline
%   %\ganttmilestone{MPC conference 2018}{6}  \ganttnewline

%   \ganttbar[name=i1]{SQP Solver}{1}{2}  \ganttnewline
%   \ganttbar[name=i2]{Input Param.}{2}{2}  \ganttnewline
%   \ganttbar[name=i3]{Code  Integration \\
%   and refactoring}{4}{5}  \ganttnewline
%   \ganttbar[name=i4]{Documentation}{3}{5}  \ganttnewline
%   \ganttbar[name=i5]{Untitled \texttt{ICLOCS} Work}{7}{32}  \ganttnewline
  
%   %% PROJECT SUPERVISION
%   \ganttgroup{Masters Project Supervision}{1}{34} \ganttnewline
  
%   \ganttbar[name=p1]{Comms. HW project }{1}{9}  \ganttnewline
%   \ganttbar[name=p2]{Untitled Comms./Control Project}{13}{21}     
%   \ganttbar[name=p3]{}{25}{33}  \ganttnewline 

%   \ganttbar[name=p4]{Crane Control Project }{1}{9} 
%   \ganttbar[name=p5]{}{13}{21}  \ganttnewline    
  
%   %% Links ICLOCS
%   \ganttlink{i1}{i3}
%   \ganttlink{i2}{i3}
%   \ganttlink{i3}{i5}
%   %% Links across Projects
%   \ganttlink{p1}{p2}
%   \ganttlink{p2}{p3}
%   \ganttlink{p4}{p5}
  
% \end{ganttchart}

\section{Publications(Papers \& Extended Abstracts)}
\nocite{*}

\bibliography{bibliography} 
\bibliographystyle{ieeetr}
\end{document}
